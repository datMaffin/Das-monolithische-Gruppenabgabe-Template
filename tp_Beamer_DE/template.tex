\documentclass{beamer}

% Lade Pakete, die Symbole hinzufügen könnten.
% Diese sollten vor den Fonteinstellungen geladen werden; 
% siehe unicode-math Dokumentation.
\usepackage{amsmath} % Setup before fonts...

\usepackage{FiraSans}
\usepackage[mathrm=sym]{unicode-math}
\setmathfont{Fira Math}
%\usepackage[tx]{plex-mono}  % possible to use instead of sourcecodepro
\usepackage[scale=1.06]{sourcecodepro}  % I feel like it fits better with fira sans
\usepackage[ngerman]{babel}
\usepackage[ngerman]{selnolig}

% Beamer Einstellungen ========================================================
\beamertemplatenavigationsymbolsempty%
\usetheme{default}

% Nummeriere Sektionen und Subsektionen
\setbeamertemplate{section in toc}[sections numbered]
\setbeamertemplate{subsection in toc}[subsections numbered]
\setbeamertemplate{footline}[frame number]

\newcommand{\backupbegin}{
   \newcounter{framenumberappendix}
   \setcounter{framenumberappendix}{\value{framenumber}}
}
\newcommand{\backupend}{
   \addtocounter{framenumberappendix}{-\value{framenumber}}
   \addtocounter{framenumber}{\value{framenumberappendix}} 
}

\AtBeginSection[] {%
    \begin{frame}
        \frametitle{\insertsectionhead}
        \tableofcontents[currentsection]
        \pdfpcnote{Gliederung}
    \end{frame}
}

% Weitere Usepackage Befehle ==================================================
\usepackage[duration=30]{pdfpc}
\usepackage{graphicx}
\usepackage{listings}

% Mein Colorscheme ============================================================
\definecolor{my-blue}{rgb}{0.0392,0.333,0.5804}
\definecolor{my-green}{rgb}{0,0.7216,0}
\definecolor{my-red}{rgb}{0.902,0,0}
\definecolor{my-orange}{rgb}{0.902,0.4314,0}
\definecolor{my-magenta}{rgb}{0.702,0,0.4392}

% LstListing-Formate ==========================================================
\lstdefinestyle{cpp}{
    language=C++,
    basicstyle=\small\ttfamily,
    frame=tb,
    xleftmargin=\parindent,
    keywordstyle=\color{my-blue}\textbf,
    stringstyle=\color{my-red},
    commentstyle=\color{my-green}\textit,
    morecomment=[l][\color{my-magenta}]{\#},
    framexleftmargin=5pt,
    framexrightmargin=5pt,
    framextopmargin=5pt,
    framexbottommargin=5pt,
    literate={~}{$\sim$}1,
    morekeywords={concept, requires}
}


% Titel Slide =================================================================
\title{Titelname}
\subtitle{Untertitel}
\author{Marvin Dostal}
\institute{{\bf Betreuer:} Betreuer Name}
\date{\today}

% Start Document ==============================================================
\begin{document}

\frame{\titlepage}

\begin{frame}
    \frametitle{Gliederung}
    \tableofcontents[hideallsubsections]
    \clearpage

    \pdfpcnote{Gliederung}
    \pdfpcnote{1 Einleitung}
    \pdfpcnote{2 Dünngitter}
    \pdfpcnote{3 MPI}
    \pdfpcnote{4 Combigrid Framework}
    \pdfpcnote{5 Lastbalancierung}
    \pdfpcnote{6 Aussichten}
\end{frame}

\section{Einleitung}
\subsection{C-Style}

\begin{frame}[fragile] % Begin Frame ------------------------------------------
    \frametitle{\secname: \subsecname}

    \pdfpcnote{Einfachste Art der Typenlöschung...}
    \pdfpcnote{}
    \pdfpcnote{Kombinationsmöglichkeiten schon sichtbar!}

    \begin{block}{Cast zu einem d \texttt{dvoidP}-Pointer}
        \begin{lstlisting}[style=cpp, numbers=left]
// random comment
YT_1300F millenium_falcon;
Constitution uss_enterprise;

void *arr[2];
arr[0] = (void*) &millenium_falcon;
arr[1] = (void*) &uss_enterprise;
        \end{lstlisting}

        \begin{itemize}
            \item Typ wird durch den Cast gelöscht
                \pause
            \item Kombiniationsmöglichkeiten durch Type Erasure erkennbar
                \begin{itemize}
                    \item Zwei (völlig) unabhängige Objekte können in ein Array 
                        zusammengefasst werden
                \end{itemize}
        \end{itemize}
    \end{block}

\end{frame} 

\begin{frame}
    \begin{block}{Etwas Mathematisches}
        \begin{itemize}
            \item $a = b + 1$
            \item $\alpha \beta = \sum_i i$
        \end{itemize}
    \end{block}
\end{frame}

\begin{frame}
    \begin{block}{Etwas italics}
        \begin{itemize}
            \item Hier ist etwas \textit{italic}. 
            \item \textit{Mehr italics!}
        \end{itemize}
    \end{block}

    \begin{block}{Etwas bold}
        \begin{itemize}
            \item Hier ist etwas \textbf{bold}. 
            \item \textbf{Mehr bold!}
        \end{itemize}
    \end{block}
    
    \begin{block}{Etwas bold italic mono}
        \begin{itemize}
            \item Hier ist etwas \texttt{\textit{\textbf{mono bold-italic}}}. 
        \end{itemize}
    \end{block}
\end{frame}

\begin{frame}
    \begin{block}{Teste Ligaturen}
        \begin{itemize}
            \item fliegen hoffen Hofflasche
            \item \textit{fliegen hoffen}
        \end{itemize}
    \end{block}
\end{frame}
% End Frame -------------------------------------------------------------------


\end{document}
