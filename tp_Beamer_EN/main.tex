\documentclass{beamer}

\usepackage{amsmath} % Setup before fonts; see unicode-math documentation

%\usepackage[tx]{plex-mono}  % possible to use instead of sourcecodepro
%\usepackage[scale=1.06]{sourcecodepro}  % I feel like it fits better with fira sans
%\usepackage[scale=1.06]{sourceserifpro}

% Mono version of Fira; italics are only slanted. However, avoiding italics in presentation may be good anyhow...
\usepackage[lf]{FiraMono} 

\usepackage{FiraSans}
\usepackage[mathrm=sym]{unicode-math}
\setmathfont{Fira Math}
\usepackage[english]{babel}
\usepackage[english]{selnolig}

% My Beamer Settings ==========================================================
\beamertemplatenavigationsymbolsempty%
\usetheme{default}

% Number sections and subsections
\setbeamertemplate{section in toc}[sections numbered]
\setbeamertemplate{subsection in toc}[subsections numbered]
\setbeamertemplate{footline}[frame number]

\newcommand{\backupbegin}{
   \newcounter{framenumberappendix}
   \setcounter{framenumberappendix}{\value{framenumber}}
}
\newcommand{\backupend}{
   \addtocounter{framenumberappendix}{-\value{framenumber}}
   \addtocounter{framenumber}{\value{framenumberappendix}} 
}

\AtBeginSection[] {%
    \begin{frame}
        \frametitle{\insertsectionhead}
        \tableofcontents[currentsection]
        \note{Gliederung}
    \end{frame}
}

% More used packages ==========================================================

\usepackage{pgfpages} % sometimes necessary for \setbeameroption{show notes on second screen} to work
% comment out when no notes are needed
\setbeameroption{show notes on second screen}

\usepackage[duration=30]{pdfpc}
\usepackage{graphicx}
\usepackage{listings}
% ...


% listings settings ===========================================================
\definecolor{my-blue}{rgb}{0.0392,0.333,0.5804}
\definecolor{my-green}{rgb}{0,0.7216,0}
\definecolor{my-red}{rgb}{0.902,0,0}
\definecolor{my-orange}{rgb}{0.902,0.4314,0}
\definecolor{my-magenta}{rgb}{0.702,0,0.4392}

\lstdefinestyle{cpp}{
    language=C++,
    basicstyle=\small\ttfamily,
    frame=tb,
    xleftmargin=\parindent,
    keywordstyle=\color{my-blue}\textbf,
    stringstyle=\color{my-red},
    commentstyle=\color{my-green}\textit,
    morecomment=[l][\color{my-magenta}]{\#},
    framexleftmargin=5pt,
    framexrightmargin=5pt,
    framextopmargin=5pt,
    framexbottommargin=5pt,
    literate={~}{$\sim$}1,
    morekeywords={concept, requires}
}


% Titel slide =================================================================
\title{On Learning Meaningful Assert Statements for Unit Test Cases}
\subtitle{Seminar: Machine Learning for Programming}
\author{Marvin Dostal}
\institute{Student at the University of Stuttgart}
\date{\today}

% Start document ==============================================================
\begin{document}

\frame{\titlepage}

\begin{frame}
    \frametitle{Outline}
    \tableofcontents[hideallsubsections]
    \clearpage

    \note{
        Outline
        \begin{enumerate}
            \item Introduction
            \item Approach
            \item Evaluation
            \item Discussion
            \item Conclusion
        \end{enumerate}
    }
\end{frame}

\section{Outline}
\subsection{C-Style}

\begin{frame}[fragile] % Begin Frames ------------------------------------------
    \frametitle{\secname: \subsecname}

    \begin{block}{Cast to \texttt{dvoidP}-Pointer}
        \begin{lstlisting}[style=cpp, numbers=left]
// random comment
YT_1300F millenium_falcon;
Constitution uss_enterprise;

void *arr[2];
arr[0] = (void*) &millenium_falcon;
arr[1] = (void*) &uss_enterprise;
        \end{lstlisting}

        \begin{itemize}
            \item Typ deleted by cast
                \pause
            \item Possible combinations recognizeable thatnks to \textbf{Type Erasue}
                \begin{itemize}
                    \item Two (completely) independent objects are able to be included in one array.
                \end{itemize}
        \end{itemize}
    \end{block}

    \note{
        \begin{itemize}
            \item Einfachste Art der Typenlöschung...
            \item Kombinationsmöglichkeiten schon sichtbar!
        \end{itemize}
    }

\end{frame} 

\begin{frame}
    \begin{block}{Little bit of math}
        \begin{itemize}
            \item $a = b + 1$
            \item $\alpha \beta = \sum_i i$
            \item $\forall x \in \mathbb{N}\cos(x) = -\sin(x)$
        \end{itemize}
    \end{block}
\end{frame}

\begin{frame}
    \begin{block}{Some italics}
        \begin{itemize}
            \item Here is \textit{italic}. 
            \item \textit{More italics!}
        \end{itemize}
    \end{block}

    \begin{block}{Something bold}
        \begin{itemize}
            \item Here is \textbf{bold}. 
            \item \textbf{More bold!}
        \end{itemize}
    \end{block}
    
    \begin{block}{Some bold italic mono}
        \begin{itemize}
            \item Here is \texttt{\textit{\textbf{mono bold-italic}}}. 
        \end{itemize}
    \end{block}
\end{frame}

\begin{frame}
    \begin{block}{Test Ligature}
        \begin{itemize}
            \item fly Kafka
            \item \textit{fly Kafka}
        \end{itemize}
    \end{block}
\end{frame}
% End Frame -------------------------------------------------------------------


\end{document}
